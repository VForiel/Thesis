\chapter{Theoretical Foundations}
\label{ch:theory}

\section{Optical Interferometry and Nulling}

Optical interferometry combines light from multiple apertures to achieve high angular resolution. The fundamental observable is the complex visibility, related to the spatial frequency content of the source via the Van Cittert-Zernike theorem.

\subsection{The Bracewell Nuller}
Proposed by \cite{bracewell_detecting_1978}, the nulling interferometer aims to cancel the starlight by adjusting the phases of the combined beams such that they interfere destructively on axis (where the star is located) and constructively off-axis (where the planet is).
For a two-element interferometer, the transmission map $T(\theta)$ is given by:
\begin{equation}
    T(\theta) = \sin^2\left(\frac{\pi B \theta}{\lambda}\right)
\end{equation}
where $B$ is the baseline and $\lambda$ the wavelength.

\subsection{limitations of Classical Nulling}
Ideally, the null depth is infinite. In practice, it is limited by instrumental defects, primarily piston errors (phase delay differences) and intensity mismatches. These residual errors result in "stellar leakage" which drowns out the planetary signal.

\section{Kernel Nulling}

\subsection{Matrix Formalism}
An interferometric beam combiner can be described by a linear transfer matrix $\mathbf{M}$ linking the input complex amplitudes $\mathbf{E}_{in}$ to the output fields $\mathbf{E}_{out}$:
\begin{equation}
    \mathbf{E}_{out} = \mathbf{M} \cdot \mathbf{E}_{in}
\end{equation}
The detected intensities are $\mathbf{I} = |\mathbf{E}_{out}|^2$.

\subsection{Robustness to Aberrations}
\cite{martinache_kernel_2018} introduced the concept of **Kernel Nulling** (KN). By forming specific linear combinations of the output intensities (kernels), one can construct observables that are insensitive to second-order phase perturbations.
If $\mathbf{K}$ is the kernel matrix such that $\mathbf{K} \cdot \mathbf{A} = 0$ (where $\mathbf{A}$ describes the first-order response to phase errors), then the kernel-nulls $\mathbf{k} = \mathbf{K} \cdot \mathbf{I}$ are robust against small phase ocillations, stabilizing the null depth.

\section{Photonic Beam Combiners}

Integrated photonics offers a compact and stable platform for interferometry.

\subsection{Multi-Mode Interferometers (MMI)}
MMI devices are based on the self-imaging principle in multi-mode waveguides \citep{soldano1995optical}. Light injected into a wide multimode region excites multiple modes which propagate with different phase velocities. At specific distances along the propagation axis, these modes interfere to reproduce the input field or form multiple images.
For a $N \times M$ coupler, the transfer matrix is determined by the device geometry, enabling the design of complex mixing functions required for Kernel Nulling.

\section{Tunable Kernel Nulling}

To maximize the scientific yield, the beam combiner must be adaptable. By integrating active phase shifters (e.g., thermal heaters) on the photonic chip, one can effectively "tune" the transfer matrix $\mathbf{M}$ in real-time. This allows for:
\begin{itemize}
    \item Calibrating fabrication errors.
    \item Switching between different observing modes (e.g., varying the nulling baseline).
    \item Optimizing the kernel response for specific target geometries.
\end{itemize}
This thesis investigates the implementation of such tunable architectures.
